\documentclass[12pt,a4paper]{article}
\usepackage[utf8]{inputenc}
\usepackage[polish]{babel}
\usepackage[T1]{fontenc}
\usepackage{graphicx}
\usepackage{hyperref}
\usepackage[style=numeric-comp,sorting=none,backend=biber]{biblatex}
\usepackage{geometry}
\usepackage{enumitem}
\usepackage{fancyhdr}
\usepackage{titlesec}
\usepackage{float}
\usepackage{caption}
\usepackage{subcaption}

\geometry{margin=2.5cm}
\addbibresource{bibliografia.bib}

\pagestyle{fancy}
\fancyhf{}
\rhead{\thepage}
\lhead{Propozycja tematu pracy magisterskiej}

\titleformat{\section}{\normalfont\Large\bfseries}{\thesection}{1em}{}
\titleformat{\subsection}{\normalfont\large\bfseries}{\thesubsection}{1em}{}

\begin{document}

\begin{titlepage}
\centering
\vspace*{2cm}

{\LARGE\bfseries Propozycja tematu pracy magisterskiej\\[0.5cm]}
\vspace{2cm}

{\Large\bfseries Implementacja i optymalizacja algorytmów uczenia głębokiego do klasyfikacji arytmii w niskokosztowych systemach monitoringu EKG z wykorzystaniem technologii Edge AI\\[1cm]}

\vspace{3cm}

\begin{flushleft}
\large
\textbf{Student:} Łukasz Guziczak\\
\textbf{Nr albumu:} 105093\\
\textbf{Kierunek:} Informatyka\\
\textbf{Specjalność:} Sztuczna inteligencja i uczenie maszynowe\\[1cm]
\textbf{Promotor:} dr inż. Piotr Ciskowski\\
\end{flushleft}

\vfill
{\large Uniwersytet WSB Merito Wrocław\\[0.5cm]
Data: \today}

\end{titlepage}

\tableofcontents
\newpage

\section{Temat pracy}

\textbf{PL:} Implementacja i optymalizacja algorytmów uczenia głębokiego do klasyfikacji arytmii w niskokosztowych systemach monitoringu EKG z wykorzystaniem technologii Edge AI

\textbf{EN:} Implementation and Optimization of Deep Learning Algorithms for Arrhythmia Classification in Low-Cost ECG Monitoring Systems Using Edge AI Technology

\section{Uzasadnienie wyboru tematu i cel pracy}

\subsection{Motywacja}

Choroby układu krążenia stanowią główną przyczynę zgonów globalnie. Dostęp do profesjonalnej diagnostyki EKG jest ograniczony ze względu na:
\begin{itemize}
    \item Wysokie koszty profesjonalnych urządzeń (>10 000 PLN)
    \item Brak wykwalifikowanego personelu w regionach oddalonych
    \item Konieczność fizycznej wizyty w placówce medycznej
\end{itemize}

Projekt proponuje rozwiązanie wykorzystujące:
\begin{itemize}
    \item Tani moduł AD8232 (<20 PLN) + mikrokontroler RP2040 (<10 PLN)
    \item Algorytmy deep learning do automatycznej analizy
    \item Przetwarzanie Edge AI eliminujące potrzebę internetu
\end{itemize}

Autor opracował działający prototyp systemu z oprogramowaniem dostępnym w repozytorium open-source: \textbf{\url{https://github.com/guziczak/ad8232}}. Repozytorium zawiera:
\begin{itemize}
    \item \texttt{main\_scaled.py} -- firmware MicroPython wgrywany na mikrokontroler RP2040
    \item \texttt{visualizer\_scaled.py} -- aplikacja PC do wizualizacji sygnału EKG w czasie rzeczywistym
\end{itemize}

\subsection{Cel główny}

Opracowanie kompletnego systemu monitoringu EKG wykorzystującego uczenie głębokie do automatycznej klasyfikacji arytmii, działającego w czasie rzeczywistym na urządzeniach embedded o ograniczonych zasobach.

\subsection{Cele szczegółowe}

\begin{enumerate}
    \item \textbf{Walidacja sprzętowa:} Zbadanie możliwości diagnostycznych układu AD8232 w porównaniu z profesjonalnym EKG
    \item \textbf{Rozwój modeli AI:} Implementacja i porównanie architektur CNN, LSTM, CAT-Net dla klasyfikacji arytmii
    \item \textbf{Optymalizacja dla Edge:} Kompresja modeli (kwantyzacja, pruning) do rozmiaru <500KB
    \item \textbf{Walidacja kliniczna:} Osiągnięcie dokładności >90\% na datasetach MIT-BIH \cite{mitbih1997} i PTB-XL \cite{ptbxl2020}
\end{enumerate}

\section{Hipotezy badawcze}

\begin{enumerate}
    \item[\textbf{H1:}] Sygnał z układu AD8232 po preprocessingu umożliwia detekcję podstawowych arytmii (AF, PVC, bradykardia, tachykardia) z czułością >85\%
    
    \item[\textbf{H2:}] Architektury CNN dedykowane sygnałom 1D osiągają lepszy stosunek dokładność/złożoność niż LSTM dla single-lead EKG
    
    \item[\textbf{H3:}] Kwantyzacja INT8 i pruning redukują rozmiar modelu o >75\% przy spadku dokładności <5\%
    
    \item[\textbf{H4:}] Edge AI zapewnia latencję <100ms, umożliwiając real-time monitoring bez połączenia z chmurą
\end{enumerate}

\section{Metodologia badawcza}

\subsection{Architektury modeli}

\begin{enumerate}
    \item \textbf{CNN 1D} - Adaptacja MobileNet dla sygnałów czasowych
    \item \textbf{LSTM z uwagą} - Analiza długoterminowych zależności w EKG
    \item \textbf{CAT-Net} - Hybrydowa architektura Convolution-Attention-Transformer
    \item \textbf{Ensemble} - Voting classifier łączący powyższe
\end{enumerate}

\subsection{Techniki optymalizacji}

\begin{itemize}
    \item \textbf{Kwantyzacja:} Post-training quantization FP32→INT8
    \item \textbf{Pruning:} Magnitude-based i structured pruning
    \item \textbf{Knowledge Distillation:} Transfer z dużego modelu-nauczyciela
    \item \textbf{NAS:} Automatyczne wyszukiwanie optymalnej architektury
\end{itemize}

\subsection{Datasety}

\begin{itemize}
    \item \textbf{MIT-BIH Arrhythmia Database} \cite{mitbih1997} - 48 nagrań 30-minutowych, gold standard dla arytmii
    \item \textbf{PTB-XL Database} \cite{ptbxl2020} - 21,837 nagrań 12-odprowadzeniowych, największy publiczny dataset
    \item \textbf{Własne dane} - Zebrane układem AD8232 do fine-tuningu
\end{itemize}

\subsection{Metryki ewaluacji}

\begin{itemize}
    \item \textbf{Skuteczność:} Accuracy, Sensitivity, Specificity, F1-score per klasa
    \item \textbf{Wydajność:} Latencja (ms), FPS, zużycie RAM/Flash
    \item \textbf{Energia:} Pobór prądu (mA), czas pracy na baterii
\end{itemize}

\section{Plan realizacji}

\subsection{Harmonogram}

\begin{enumerate}
    \item \textbf{Październik-Listopad 2025:} Przegląd literatury, analiza state-of-the-art
    \item \textbf{Grudzień 2025:} Walidacja sprzętu AD8232, porównanie z profesjonalnym EKG
    \item \textbf{Styczeń-Marzec 2026:} Rozwój i trening modeli ML na datasetach
    \item \textbf{Kwiecień-Maj 2026:} Kompresja modeli i deployment na RP2040
    \item \textbf{Czerwiec 2026:} Testy systemu, walidacja cross-dataset
\end{enumerate}

\subsection{Kamienie milowe}

\begin{itemize}
    \item[\textbf{M1:}] Działający prototyp hardware z podstawowym preprocessingiem
    \item[\textbf{M2:}] Wytrenowane modele z accuracy >90\% na MIT-BIH
    \item[\textbf{M3:}] Model skompresowany do <500KB działający na RP2040
    \item[\textbf{M4:}] Kompletny system z dokumentacją open-source
\end{itemize}

\section{Aktualny stan projektu}

\begin{figure}[H]
    \centering
    \includegraphics[width=0.8\textwidth]{image.png}
    \caption{Obecny prototyp systemu EKG z modułem AD8232 i wizualizacją real-time}
    \label{fig:prototype}
\end{figure}

\begin{figure}[H]
    \centering
    \includegraphics[width=0.7\textwidth]{hardware_photo.jpeg}
    \caption{Prototyp hardware'u: mikrokontroler RP2040-Zero z modułem AD8232 oraz elektrodami EKG}
    \label{fig:hardware}
\end{figure}

Dotychczas zrealizowano:
\begin{itemize}
    \item Działający hardware (RP2040 + AD8232)
    \item Firmware w MicroPython (250 Hz sampling)
    \item PC software do wizualizacji real-time
\end{itemize}

Zidentyfikowane wyzwania:
\begin{itemize}
    \item Sygnał wymaga skalowania (obecnie 907x za duży)
    \item Konieczność implementacji zaawansowanego filtrowania
    \item Duże szumy przy ruchu pacjenta (artefakty ruchowe)
    \item Błędna detekcja HR przez visualizer (~150 BPM zamiast rzeczywistych) z powodu problemu skalowania
    \item Optymalizacja dla niskich zasobów RP2040 (264KB RAM)
\end{itemize}

\section{Oczekiwane rezultaty}

\subsection{Wkład naukowy}
\begin{itemize}
    \item Pierwsze systematyczne badanie AD8232 w kontekście deep learning
    \item Analiza trade-off dokładność vs zasoby dla embedded AI w medycynie
    \item Porównanie skuteczności architektur DL dla single-lead EKG
\end{itemize}

\subsection{Wkład praktyczny}
\begin{itemize}
    \item Open-source framework dla tanich systemów EKG (<50 PLN)
    \item Zwalidowany pipeline od surowych danych do klasyfikacji
    \item Dokumentacja umożliwiająca replikację projektu
\end{itemize}


\nocite{*}
\printbibliography[heading=bibintoc,title={Bibliografia podstawowa}]

\end{document}